\documentclass[11]{article}
\usepackage[margin=1in, paperwidth=8.5in, paperheight=11in]{geometry}
\usepackage{amsfonts}
\usepackage[english]{babel}
\usepackage{amsmath}
\usepackage{hyperref}
\usepackage{graphicx}
\begin{document}
\title{Minkowski space and Special Relativity}
\author{Muhammad Shatla \hspace {10mm} Doaa Atteya}
\date{\today}
\maketitle
\newpage
\tableofcontents
\newpage

\section{Introduction}
Befor Einstein introduce his postulates of special theory of relativity  the relevant mathematical foundation was already established. Actually Einstein build his theory on his professor Herman Minkowski spacetime.
In the early 1900, after michelson-moorely experimental result that the speed of light is constant, the Gallilian transformation was not able to explain this strange but true result. Scientists need a nother transformation that can explain the constant speed of light and also, the the real life newtonian mechanics. \\
For a complete treatment of relativistic space and time, Einstein's General Theory of Relativity is needed.
His special theory makes one assumption that is shown to be false in the more general case: space is linear.
In fact, many fascinating effects arise from the non-linearity of space, such as black holes and the curvature
of the Universe. But, particularly in the absence of enormous masses, space can be closely approximated
by linear means. This motivates a linear algebraic approach to special relativity. To proceed, we will put
forward the postulates of special relativity, define the mathematicl terminology necessary, motivated certain
mathematical assumptions with physical arguments, and use the power of linear algebra to deduce the Lorentz
transformations. This paper is aimed at an audience familiar with linear algebra.

\section{The three Postulates of Special Relativity}
\begin{enumerate}
\item All references frames are equivalent, or that no single reference frame is in any way special.
\item The speed of light, measured in any reference frame and in any direction, is c.
\item Equally spaced increments of space and time in one reference frame correspond to equally spaced incre-
ments of space and time in any other reference frame.
\end{enumerate}
The first postulate is know as the relativity principle. The second one is a consequence of maxwell’s equations in Electrodynamic and the third one is about the homogeneity of space.

\section{Essential Representations}

To work with the physics of relativity in a linear algebra format,it is neccessary to define a vector space involving the space and time coordinates of our perceived reality. For physical reasons, each component of
these vectors should have the same units. Given that in normal 3 dimensional space the units of any point
are units of distance, it is logical to extend the time component to a distance unit by multiplying by the speed
of light. That is, in the space-time vector space, the time like coordinate is ct, which does have the correct
units of distance. Thus, a point in space-time with the space coordinates of $x, y, z$ and time coordinate of t
may be represented as:
\[
\begin{pmatrix}
ct\\
x\\
y\\
z
\end{pmatrix}
\]
\newpage
As such, the set of all w clearly constitutes a vector space M (commonly called Minkowski Space). we need a linear transformation from M to itself that preserves the speed of light
and homogeneity of space. This linear transformation will physically represent the conversion between two
different reference frames, and mathematically will be an isomorphism from M to M.
The intuitive concept of Reference Frame is obvious; 
Mathematically, we define a reference frame to be apoint \textbf{o} in space-time withr the coordinates:

$$
0=\begin{pmatrix}
0\\
0\\
0\\
0
\end{pmatrix}
$$
and we orient our three space axes such that the unit vectors in those directions, along with th time component, are defined as following:

$$
e_0=\begin{pmatrix}
1\\
0\\
0\\
0
\end{pmatrix}
,e_1=\begin{pmatrix}
0\\
1\\
0\\
0
\end{pmatrix}
,e_2=\begin{pmatrix}
0\\
0\\
1\\
0
\end{pmatrix}
,e_3=\begin{pmatrix}
0\\
0\\
0\\
1
\end{pmatrix}
$$
The zeroth basis vector, e0, is considered time, and the other three coordinates are the traditional set.
I will call such a reference frame S. We will denote any other reference frame moving with speed v with
respect to S as S0. S0 has all the same properties of S (as ensured by Postulate 1),

\section{Minkowski Metric}

Before we get into Minkowski space, we will first talk about Minkowski metric. The formal defenition is a tensor $\eta_{\mu \nu}$ whose elements are defined by the matrix:
\[
\eta_{\mu \nu}=\begin{pmatrix}
-1&0&0&0\\
0&1&0&0\\
0&0&1&0\\
0&0&0&1
\end{pmatrix} 
\]

\noindent we take the convection that $c=-1$ and the indices $\alpha$ and $\beta$ run over values (0 1 2 3), where $x^0=t$ and $(x^1,x^2,x^3)$ are the euclidean space coordinates.

\noindent The Euclidean metric $g_{\mu \nu}=\begin{pmatrix}
1&0&0\\
0&1&0\\
0&0&1
\end{pmatrix}$ gives a line element:

\begin{align*} 
ds^2&=g_{\mu \nu}d x^{\mu} d x^{\nu}\\
&=(dx^1)^2+(dx^2)^2+(dx^3)^2
\end{align*}
while the Minkowski metric gives its relativistic generalization, the proper time:\\
\begin{align*}
d\tau^2&=\eta_{\mu \nu} d x^\mu d x^\nu\\
&=-(dx^0)^2+(dx^1)^2+(dx^2)^2+(dx^3)^2
\end{align*}

\section{Invariance of Minokowski space (spacetime)}
In normal euclidean space $R^3$ the length from a point $(x,y,z)$ to another poin $(x',y',z')$ can be calculated from the extended Phythagorean theorem \\
\begin{center} $L^2=(x'-x)^2+(y'-y)^2+(z'-z)^2$
\end{center}
Now, In Minkowski space the interval between two events is invariant despite the frame of reference we are measures from.
So, the interval \textbf{s} between point $(x,y,z,ct)$ and pont %(x',y',z',ct')$ can be calculated 
\begin{center}
$s^2=(x'-x)^2+(y'-y)^2+(z'-z)^2-(ct'-ct)^2$
\end{center}
As a result,provided that the lorentz transformation is linear, the quadratic form can be used to define a bilinear form:
\begin{align*}
u.v&= \pm[c^2tt'-xx'-yy'-zz']\\
&=u^T[\eta]v\\
&=\eta(u,v)
\end{align*}
a bilinear form on a vector space V is a bilinear map V × V → K, where K is the field of scalars. In other words, a bilinear form is a function $B : V \times V \mapsto K$ that is linear in each argument separately:\\
\begin{align*}
B(u+v,w)&=b(u,w)+B(v,w) and B(\lambda u,v)=\lambda B(u,v)\\
B(u,v+w)&=b(u,v)+B(u,w) and B(u,\lambda v)=\lambda B(u,v)
\end{align*}

\section{The Formal Definition of Minkowski space}
Minkowski space is a combination of three-dimensional Euclidean space and time into a four-dimensional manifold where the spacetime interval between any two events is independent of the inertial frame of reference in which they are recorded.
\begin{center}
\includegraphics[scale=.5]{capture.png}
\end{center}



\section{Standard Basis}
A standard basis for Minkowski space is a set of four mutually orthogonal vectors { e0, e1, e2, e3 } such that:\\
\begin{center}
$-\eta(e_0,e_0)=\eta(e_1,e_1)=\eta(e_2,e_2)=\eta(e_3,e_3)=1$
\end{center}
More compactly:
\begin{center}
$\eta(e_{\mu},e_{\nu})=\eta_{\mu \nu}$
\end{center} 

Relative to a standard basis, the components of a vector v are written $(v^0, v^1, v^2, v^3)$ where the Einstein notation is used to write v = vμeμ. The component $v^0$ is called the timelike component of v while the other three components are called the spatial components. The spatial components of a 4-vector v may be identified with a 3-vector $v = (v^1, v^2, v^3)$.

\section{References}

Spence Friedberg, Insel. Linear Algebra, 4/E. Pearson, 2003.\\

\noindent Stover, Christopher, Weisstein and Eric W. (Minkowski Space) Retrieved From: http://mathworld.wolfram.com/MinkowskiSpace.html\\

\noindent Wolfgang Rindler. Introduction to Special Relativity. Oxford University Press, 1982.




\end{document}