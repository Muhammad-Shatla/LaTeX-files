% Copyright 2004 by Till Tantau <tantau@users.sourceforge.net>.
%
% In principle, this file can be redistributed and/or modified under
% the terms of the GNU Public License, version 2.
%
% However, this file is supposed to be a template to be modified
% for your own needs. For this reason, if you use this file as a
% template and not specifically distribute it as part of a another
% package/program, I grant the extra permission to freely copy and
% modify this file as you see fit and even to delete this copyright
% notice. 

\documentclass{beamer}
% Replace the \documentclass declaration above
% with the following two lines to typeset your 
% lecture notes as a handout:
%\documentclass{article}
%\usepackage{beamerarticle}


% There are many different themes available for Beamer. A comprehensive
% list with examples is given here:
% http://deic.uab.es/~iblanes/beamer_gallery/index_by_theme.html
% You can uncomment the themes below if you would like to use a different
% one:
%\usetheme{AnnArbor}
%\usetheme{Antibes}
%\usetheme{Bergen}
%\usetheme{Berkeley}
%\usetheme{Berlin}
%\usetheme{Boadilla}
%\usetheme{boxes}
%\usetheme{CambridgeUS}
%\usetheme{Copenhagen}
%\usetheme{Darmstadt}
%\usetheme{default}
%\usetheme{Frankfurt}
\usetheme{Goettingen}
%\usetheme{Hannover}
%\usetheme{Ilmenau}
%\usetheme{JuanLesPins}
%\usetheme{Luebeck}
%\usetheme{Madrid}
%\usetheme{Malmoe}
%\usetheme{Marburg}
%\usetheme{Montpellier}
%\usetheme{PaloAlto}
%\usetheme{Pittsburgh}
%\usetheme{Rochester}
%\usetheme{Singapore}
%\usetheme{Szeged}
%\usetheme{Warsaw}

\title{Minkowski Space}

% A subtitle is optional and this may be deleted
\subtitle{Linear Algebraic prespective}

\author{Muhammad Shatla\inst{1} \and Doaa Atteya\inst{2}}
% - Give the names in the same order as the appear in the paper.
% - Use the \inst{?} command only if the authors have different
%   affiliation.

\institute[University of Science And Technology at Zewailcity] % (optional, but mostly needed)
{
  \inst{1}%
  Department of Mathematics\\
  University of University of Science And Technology at Zewailcity
}
% - Use the \inst command only if there are several affiliations.
% - Keep it simple, no one is interested in your street address.

\date{Project Presentation, 2017}
% - Either use conference name or its abbreviation.
% - Not really informative to the audience, more for people (including
%   yourself) who are reading the slides online

\subject{Linear Algebra and Vector Geometry}
% This is only inserted into the PDF information catalog. Can be left
% out. 

% If you have a file called "university-logo-filename.xxx", where xxx
% is a graphic format that can be processed by latex or pdflatex,
% resp., then you can add a logo as follows:

% \pgfdeclareimage[height=0.5cm]{university-logo}{university-logo-filename}
% \logo{\pgfuseimage{university-logo}}

% Delete this, if you do not want the table of contents to pop up at
% the beginning of each subsection:
\AtBeginSubsection[]
{
  \begin{frame}<beamer>{Outline}
    \tableofcontents[currentsection,currentsubsection]
  \end{frame}
}

% Let's get started
\begin{document}

\begin{frame}
  \titlepage
\end{frame}

\begin{frame}{Outline}
  \tableofcontents
  % You might wish to add the option [pausesections]
\end{frame}

% Section and subsections will appear in the presentation overview
% and table of contents.
\section{Introduction}
\begin{frame}{Introduction}{A Glimps of History}
  \begin{itemize}
  \item {
   Befor Einstein introduce his postulates of special theory of relativity the relevant mathematical foundation
was already established.
  }
  \item {
    Actually, Einstein build his theory on his professor Herman Minkowski spacetime.
  }
  \item {the Gallilian
transformation was not able to explain the constancy of the speed of light.}
\item{Scientists need a nother transformation
that can explain the constant speed of light and also, the the real life newtonian mechanics.}
\item{In the absence of enormous masses, space can be closely approximated by
linear means.}
\item{This motivates a linear algebraic approach to special relativity. To proceed, I will put forward
the postulates of special relativity}
  \end{itemize}
\end{frame}



\section{The three Postulates of Special Relativity}
\begin{frame}{The three Postulates of Special Relativity}
\begin{enumerate}
\item All references frames are equivalent, or that no single reference frame is in any way special.

\item The speed of light, measured in any reference frame and in any direction, is c.
\item Equally spaced increments of space and time in one reference frame correspond to equally spaced increments of space and time in any other reference frame.
\end{enumerate}
\end{frame}

\section{Essential Mathematical Representation}
\begin{frame}{Essential Mathematical Representation}

\begin{itemize}
\item Given that in normal 3 dimensional space the units of any point
are units of distance, it is logical to extend the time component to a distance unit by multiplying by the speed
of light.
\item in the space-time vector space, the time like coordinate is ct, which does have the correct
units of distance.
\item a point in space-time with the space coordinates of x; y; z and time coordinate of t
may be represented as:
\begin{center}
$\begin{pmatrix}
ct\\
x\\
y\\
z
\end{pmatrix}$
\end{center}
\end{itemize}
\end{frame}

\begin{frame}
\begin{itemize}
\item Mathematically, we need a reference frame to be a point o in space-time with the coordinates:
\begin{center}
$$
0=\begin{pmatrix}
0\\
0\\
0\\
0
\end{pmatrix}
$$
\end{center}

\item we
need a linear transformation from M to itself that preserves the speed of light and homogeneity of space.
\item mathematically will be an isomorphism from M to M.
\item the new unit vectors:
$$
e_0=\begin{pmatrix}
1\\
0\\
0\\
0
\end{pmatrix}
,e_1=\begin{pmatrix}
0\\
1\\
0\\
0
\end{pmatrix}
,e_2=\begin{pmatrix}
0\\
0\\
1\\
0
\end{pmatrix}
,e_3=\begin{pmatrix}
0\\
0\\
0\\
1
\end{pmatrix}
$$
\end{itemize}
\end{frame}

 \section{Minkowski Metric}
 \begin{frame}{Minkowski Metric}
 \begin{itemize}
  
 \item The formal defenition is a tensor $\eta_{\mu \nu}$ whose elements are defined by the matrix:
\[
\eta_{\mu \nu}=\begin{pmatrix}
-1&0&0&0\\
0&1&0&0\\
0&0&1&0\\
0&0&0&1
\end{pmatrix} 
\]
\item we take the convection that $c=-1$ and the indices $\alpha$ and $\beta$ run over values (0 1 2 3).
\item where $x^0=t$ and $(x^1,x^2,x^3)$ are the euclidean space coordinates.

 \end{itemize}
 \end{frame}
 
 
 \begin{frame}
 \begin{itemize}
 \item The Euclidean metric $g_{\mu \nu}=\begin{pmatrix}
1&0&0\\
0&1&0\\
0&0&1
\end{pmatrix}$ gives a line element:
\begin{align*} 
ds^2&=g_{\mu \nu}d x^{\mu} d x^{\nu}\\
&=(dx^1)^2+(dx^2)^2+(dx^3)^2
\end{align*}
while the Minkowski metric gives its relativistic generalization, the proper time:\\
\begin{align*}
d\tau^2&=\eta_{\mu \nu} d x^\mu d x^\nu\\
&=-(dx^0)^2+(dx^1)^2+(dx^2)^2+(dx^3)^2
\end{align*}
 \end{itemize}
 \end{frame}
 
\section{Invariance of Minokowski space (spacetime)}
\begin{frame}
\begin{itemize}
\item In normal euclidean space $R^3$ the length from a point $(x,y,z)$ to another poin $(x',y',z')$ can be calculated from the extended Phythagorean theorem
\begin{center} $L^2=(x'-x)^2+(y'-y)^2+(z'-z)^2$
\end{center}
\item Now, In Minkowski space the interval between two events is invariant despite the frame of reference we are measures from.
So, the interval \textbf{s} between point $(x,y,z,ct)$ and pont %(x',y',z',ct')$ can be calculated 
\begin{center}
$s^2=(x'-x)^2+(y'-y)^2+(z'-z)^2-(ct'-ct)^2$
\end{center}
\item As a result,provided that the lorentz transformation is linear, the quadratic form can be used to define a bilinear form:
\begin{align*}
u.v&= \pm[c^2tt'-xx'-yy'-zz']\\
&=u^T[\eta]v\\
&=\eta(u,v)
\end{align*}
\end{itemize}
\end{frame}

\section{The Formal Definition of Minkowski space}
\begin{frame}{The Formal Definition of Minkowski space}
Minkowski space is a combination of three-dimensional Euclidean space and time into a four-dimensional manifold where the spacetime interval between any two events is independent of the inertial frame of reference in which they are recorded.
\begin{center}
\includegraphics[scale=.5]{capture.png}
\end{center}
\end{frame}

\section{Standard Basis}
\begin{frame}{Standard Basis}
A standard basis for Minkowski space is a set of four mutually orthogonal vectors { e0, e1, e2, e3 } such that:\\
\begin{center}
$-\eta(e_0,e_0)=\eta(e_1,e_1)=\eta(e_2,e_2)=\eta(e_3,e_3)=1$
\end{center}
More compactly:
\begin{center}
$\eta(e_{\mu},e_{\nu})=\eta_{\mu \nu}$
\end{center} 
\end{frame}

\begin{frame}
\begin{itemize}
\item Relative to a standard basis, the components of a vector v are written $(v^0, v^1, v^2, v^3)$ where the Einstein notation is used to write v = vμeμ. \\
\item The component $v^0$ is called the timelike component of v while the other three components are called the spatial components. 
\item The spatial components of a 4-vector v may be identified with a 3-vector $v = (v^1, v^2, v^3)$.
\end{itemize}
\end{frame}




\appendix
\section<presentation>*{\appendixname}
\subsection<presentation>*{For Further Reading}

\begin{frame}[allowframebreaks]
  \frametitle<presentation>{For Further Reading}
    
  \begin{thebibliography}{10}
    
  \beamertemplatebookbibitems
  % Start with overview books.

  \bibitem{Spence Friedberg 2003}
    S. Friedberg.
    \newblock {\em Insel. Linear Algebra, 4/E}.
    \newblock Spence Friedberg, 2003.
 
    
  \beamertemplatearticlebibitems
  % Followed by interesting articles. Keep the list short. 

  \bibitem{Stover, Christopher, Weisstein and Eric W.}
    Stover, Christopher, Weisstein and Eric W.
    \newblock Minkowski Space
    \newblock {\url http://mathworld.wolfram.com/MinkowskiSpace.html\\}
    \beamertemplatearticlebibitems
   
    \bibitem{Wolfgang Rindler 1984}
    \newblock {\em Introduction to Special Relativity}
    \newblock Oxford University Press
  \end{thebibliography}
\end{frame}

\end{document}


