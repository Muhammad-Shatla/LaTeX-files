\documentclass[11]{article}

\usepackage[margin=1in, paperwidth=8.5in, paperheight=11in]{geometry}
\usepackage{amsfonts}
\usepackage[english]{babel}
\usepackage{hyperref}
\usepackage{graphicx}
\begin{document}
\tableofcontents
\title{Gravitaional Waves: Properties and Detection}
\author{Muhammad Shatla}
\date{\today}
\maketitle

\section{Introduction}
\setlength{\parindent}{5ex}
According to the theory of general relativity, Albert Einstein predicted the presence of different types of waves that are different from electromagnetic radiation or any other known waves; he called them gravitational waves. In fact, these waves were a direct consequence of one of the solutions of Einstein field equations; the field equations produced a non-linear system of partial differential equations and Einstein was not able to solve this system simultaneously so, he made a linearization of this system and the resultant solution was a wave equation (Bieri L., Garnkle D., Yunes N., 2017). After 10 years, Robinson and Trautman produced the first family of explicit solutions of the non-linear system of Einstein field equations; the solution was quite similar to the linearized solution. Unfortunately, the actual proof of gravitational waves was not confirmed till 1974 after Einstein death.
   
\vspace{5mm}
   In 1974, two astronomers operating at the Arecibo Radio Observatory in Puerto Rico found a binary pulsar orbiting each other. According to General Relativity, this system of immensely dense and massive stars is supposed to emit gravitational radiation. They started to measure the duration of their orbits and how it changes through the year. Following 8 years of perceptions, they found that the two stars are getting nearer to each other unequivocally at a similar rate anticipated by general relativity. This occasion has now been observed for more than 40 years and the determined changes inside the orbit concur so appropriately with general relativity.
   
\vspace{5mm}
   Astronomers have considered the timing of pulsar radio emanations and found similar results, guaranteeing the presence of gravitational waves. in any case, those affirmations had for the most part come scientifically and now not through real physical contact till September 14, 2015, while LIGO, out of the blue, physically detected bends in spacetime itself because of passing gravitational waves produced a collision of two black holes very nearly 1.3 billion light years away. LIGO and its revelation will go down in records as one of the best human scientific accomplishments. while the sources of gravitational waves might be uncommonly powerful, by the point the waves arrive the Earth they become too weak to be detected by even the most sensitive devices made by humans.
   
\vspace{5mm}
   Detecting and studying the facts carried by gravitational waves will allow us to examine the Universe in a way by no means feasible. it will open a brand-new window of examine at the Universe, provide us a deeper expertise of those cataclysmic events, and research in physics, astronomy, and astrophysics.

\vspace{5mm}
   In this article, general properties of gravitational waves will be discussed; The main theme will focus on why the gravitational waves are considered waves, what are the similarities and differences between gravitational radiation and electromagnetic radiation, why gravitational waves have a fundamental advantage over electromagnetic waves. In addition, how gravitational waves are produced and what are the cataclysmic cosmic events that can produce such waves. At last, what are the technologies used to detect gravitational waves and the types of detectors.
   
   \section{The nature of Gravitational Waves}
   
   In order to understand how gravitational wavs are produced, we should speak firstly about general relativity and the curvature of space time such that gravity manifests itself as a curvature of spacetime. GR explains gravitation as a consequence of the curvature of spacetime, while in turn spacetime curvature is a consequence of the presence of matter. Spacetime curvature affects the movement of matter, which reciprocally determines the geometric properties and evolution of spacetime.
GR is a generalization of special relativity (SR), in which Einstein set out to formulate
the laws of physics in such a way that they be valid in all inertial reference
frames – i.e. all frames in which Newton’s first and second laws of motion hold
– independently of their relative motion.
\section{The wave properties of gravitational radiation}

\begin{center}
\includegraphics[scale=1]{gwave1.png}
\includegraphics[scale=1]{gwave2.png}
\end{center}
At first, the influence of the gravitational waves shows itself in the way that the distances between the particles are changing over time. In the simple pictures here, there are two distinct possibilities: Sometimes, the gravitational wave stretches all vertical distances between particles and, at the same time, squeezes all horizontal distances. At other times, all horizontal distances are stretched while all vertical distances are squeezed (Pössel, 2006). As always in such illustrations, the stretching has been exaggerated to make it visible to the naked eye - in reality, the stretching is more than a trillion billion times less pronounced. In order to produce this pattern, the gravitational wave must be travelling at a right angle to the image plane, either directly towards or directly away from the viewer.

\vspace{5mm}
Secondly, gravitational waves are not electromagnetic radiation. they are a totally special phenomenon, carry information about cosmic events that isn't carried by electromagnetic radiation. Colliding black holes, for instance, emit very little electromagnetic radiation, but the gravitational waves they emit will cause them to shine brightly (Pössel, 2006). More importantly, since gravitational waves have interaction very weakly with matter (not like electromagnetic radiation), they travel through the universe roughly unimpeded giving us a clear view of the gravitational-wave Universe. With this new method of analysis of astrophysical events, gravitational waves will surely establish a new area of astronomy, providing astronomers and different scientists with previously unattainable details about space and time.

\vspace{5mm}
Furthermore, Gravitational waves, once they are generated, propagate almost unimpeded (Kokkotas K., 2002). Indeed, it has been proven that they are even harder to stop than neutrinos. The only significant change they suffer as they propagate is the decrease in amplitude while they travel away from their source, and the redshift they feel (cosmological, gravitational or Doppler), as is the case for electromagnetic waves. There are other factors that roughly affect the gravitational waveforms, for instance, interstellar or intergalactic matter absorption; this matter intervenes between the observer and the source, this absorption is extremely weak  (actually, the extremely weak coupling of gravitational waves with matter is the main reason that gravitational waves have not been observed) (Kokkotas K., 2002). gravitational waves scattering is also unimportant from a practical sense; however, they may have been essential during the early stages of the universe. Gravitational waves can be focused by strong gravitational fields and can be diffracted, exactly as it happens with the electromagnetic waves.

\vspace{5mm}
Gravitational waves carry energy and cause a deformation of spacetime (Kokkotas K., 2002). The energy flux has all the properties one would anticipate by analogy with electromagnetic waves: (a) it is conserved (the amplitude dies out as 1/r, the flux as 1/r2), (b) detectors can absorb it, and (c) like any other energy source, GWs can cause a curvature of spacetime. As an example, by using the above relation, we will estimate the energy flux in gravitational waves from the collapse of the core of a supernova to create a (10 Mass of the sun) black hole at 50-million-light-years (15 Mpc) from the earth (Kokkotas K., 2002). A conservative estimate of the amplitude of the waves on earth is of the order of 10−22 (at a frequency of about 1kHz). This corresponds to a flux of about 3 ergs/cm2 sec. This is an enormous amount of energy flux and is about ten orders of magnitude larger than the observed energy flux in electromagnetic waves (Kokkotas K., 2002).
\section{The generation of Gravitational waves}
Just like the production of electromagnetic radiation, gravitational waves are produced when a very massive object is accelerating resulting in an extremely powerful disturbance of spacetime curvature around it; there are several examples on this phenomenon; the first example is two neutron stars orbiting each other. They get closer gradually till coalescing together in one object. During this event, they lose energy in the form of gravitational radiation leading to decrease in the total energy of the system; This results in gradual decrease in the diameter of their orbit and eventually ends with their merging. Another example is merging of two black holes and emitting gravitational radiation that has been detected here on earth by two separate interferometers, one in Washington and the other in Livingston, both in USA.
\section{Astronomical Sources of Gravitational Waves}
The detectability of gravitational waves sources depends on three parameters: their intrinsic gravitational wave luminosity, their event rate, and their distance from the Earth (Hendry M., 2007). In fact, the quadrupole formula can roughly estimate the luminosity. Even though there are certain restrictions in its applicability (weak field, slow motion), it provides a reasonable estimate of gravitational flux on earth. Additionally, the gravitational luminosity is extrapolated from observations of electromagnetic radiation of a certain event such as a binary system of neutron stars. However, there are sources that are gravitationally luminous, but we do not have electromagnetic observation for it. Finally, the GWs amplitude decreases as one over the distance to the source. Thus, a signal from a supernova explosion might be clearly detectable if the event takes place in our galaxy (2-3 events per century), but it is highly unlikely to be detected if the supernova explosion occurs at far greater distances, of order 100 Mpc, where the event rate is high and at least a few events per day take place (Hendry M., 2007).\\
All three factors have to be considered when discussing sources of gravitational waves. the frequency of gravitational waves is proportional to the square root of the mean density of the emitting system; this is approximately true for any gravitating system. For example, neutron stars usually have masses around 1.4 (Mass of the sun) and radii in the order of 10 km; thus, if we use these numbers in the relation (f =sqrt(GM/R3)) we find that an oscillating neutron star will emit gravitational waves primarily at frequencies of 2-3 kHz. By analogy, a black-hole with a mass 100 times Mass of the sun, will have a radius of 300 km and the natural oscillation frequency will be 100 Hz. Finally, for a binary system, Kepler’s law provides a direct and accurate estimation of the frequency of the emitted gravitational waves. For two 1.4 Mass of the sun neutron stars orbiting around each other at a distance of 160 Km, Kepler’s law predicts an orbital frequency of 50 Hz, which leads to an observed gravitational wave frequency of 100 Hz.

\subsection{Radiation from gravitational collapse}
Type II supernovae are associated with the core collapse of a massive star which leaves behind a rapidly rotating neutron star or a black hole, if the core has mass of > 2−3 Mass of the sun. The typical signal from such an explosion is broadband and peaked at around 1 kHz. Detection of such a signal was the goal of detector development over the last three decades. However, this process is not completely understood; As, an exact spherical collapse will not generate gravitational radiation Because the GW amplitude depends on the kinetic energy of non-spherical motion, Knowing the kinetic energy is very important.

\subsection{Radiation from binary systems}
Binary systems are the best sources of gravitational waves because they emit enormous amounts of gravitational radiation, and for a given system we know exactly what is the amplitude and frequency of the gravitational waves in terms of the masses of the two bodies and their separation. If a binary system emits detectable gravitational radiation in the bandwidth of our detectors, we can easily identify the parameters of the system.

\subsection{Radiation from spinning neutron stars}
A perfectly axisymmetric rotating body does not emit any gravitational radiation.
Neutron stars are axisymmetric configurations, but small deviations cannot be ruled
out. Irregularities in the crust (perhaps imprinted at the time of crust formation),
strains that have built up as the stars have spun down, off-axis magnetic fields,
and/or accretion could distort the axis symmetry. A bump that might be created at
the surface of a neutron star spinning with frequency (f) will produce gravitational
waves at a frequency of (2f) and such a neutron star will be a weak but continuous and almost monochromatic source of gravitational waves. The radiated energy comes at the expense of the rotational energy of the star, which leads to a spin down of the star. If gravitational wave emission contributes considerably to the observed spin down of pulsars, then we can estimate the amount of the emitted energy.

\vspace{5mm}
\section{Technologies used for detecting Gravitational waves}
The detection of gravitational waves is a worldwide endeavor. There are many technological and engineering achievements are done specifically for this task.
This a list for the currently used detectors (Krishnan B., Aulbert C., 2010)

\subsection{Interferometric Detectors}
They are gravitational wave detector that utilizes interference between light waves to detect minute changes in distance effected by a passing gravitational wave.
\begin{enumerate}
\item GEO600 has an arm-length of 600m and is located in Ruthe, Germany (near Hannover). It is operated by the Hannover branch of the Albert-Einstein-Institute. The GEO collaboration includes the University of Glasgow, Cardiff University, and the AEI, as well as the University of Birmingham and the University of the Balearics.

\item LIGO is an abbreviation for "Laser Interferometer Gravitational Wave Observatory". The LIGO project consists of three interferometric detectors, two with arm lengths of 4 kilometers and one with arm length of 2 kilometers. They are located in Hanford, Washington, and Livingston, Louisiana.

\item Virgo is another European project, a collaboration between French and Italian physicists who have built a detector with an arm-length of 3 kilometers near Pisa in Italy. At the moment, Virgo is still being tested.

\item TAMA 300 is a detector with 300 metres arm-length, located in Tokyo. It is meant to serve as a prototype for the LCGT project, an advanced kilometre scale interferometric detector planned to be built in the Kamioka mines in Japan.

\item AIGO, the "Australian International Gravitational Observatory" is an 80 metre prototype for developing advanced interferometric techniques. A larger interferometer is in the planning stage.

\item eLISA (evolved Laser Interferometer Space Antenna) is an ESA project for a space-based interferometric gravitational wave detector. eLISA consists of three spacecrafts in orbit around the Sun moving in the formation of a near-equilateral triangle whose sides are 1 million kilometers long. eLISA is currently scheduled to fly in 2022.
\end{enumerate}

\subsection{Resonance Detectors}
The second kind of detectors are the so-called resonant detectors. The center-piece of such a detector is a solid metal object. Passing gravitational waves make that central test-mass oscillate, and these oscillations can be measured and amplified to detect the gravitational waves. Currently, there are four groups looking for gravitational waves with the help of metal cylinders with masses of about one ton:
\begin{enumerate}

\item NAUTILUS is set up in Rome.

\item EXPLORER is another experiment of the Rome group; the detector itself, however, is located in the research center CERN in Geneva.

\item ALLEGRO, short for "A Louisiana Low temperature Experiment and Gravitational wave Observatory", and the experiment itself, true to its name, is located at Louisiana State University in the US.

\item AURIGA, set up in Padova, Italy.

\item NIOBE in Perth, Australia.
\end{enumerate}

\section{Conclusion}
No doubt, the detection of gravitational waves will remain on of the greatest discoveries in human history; It will give us a new perspective while looking at our universe. In the past, man used to direct his telescope to see the wonders of the universe depending on what nature gave him of being able to see through electromagnetic radiation; Now, a new era is started with a new kind of waves that depend on another force in nature: Gravity. With the appropriate instruments and state-of-the art technologies, we are able now not only to look at what has been considered not visible but also, be able to know what has been happened to these entities in the past as well.

\newpage

\section{References}
      
     \setlength{\parindent}{5ex} 
     
     \indent Bieri L., Garnkle D., Yunes N. (2017). Gravitational Waves and Their Mathematics. The AMS NOTICES. 64(07), 693-707.\\

      Kokkotas K. (2002). the Encyclopedia of Physical Science and Technology (3rd Ed.). Greece: Academic press.\\

      Krishnan B., Aulbert C.(2010). Listening posts around the globe. Einstein online. 04(2010).Retrieved from: \url{http://www.einstein-online.info/spotlights/gw_detectors.html#author}\\

      Pössel, (2006).  The wave nature of simple gravitational waves. Einstein online. 2(2006). Retrieved from: \url{http://www.einstein-online.info/spotlights/gw_waves.html#author}\\



\end{document}