\documentclass[12]{article}
\usepackage[utf8]{inputenc}
\usepackage{amsmath}
\usepackage{amssymb}
\usepackage{graphicx}
\usepackage{graphics}
\usepackage{lmodern}
\usepackage{subfigure}
\usepackage{url}
\usepackage{flafter}
\usepackage{tikz}
\usepackage[T1]{fontenc}
\usepackage[american]{circuitikz}
\usepackage{pgfplots}

\begin{document}

\title{Lecture 11}
\maketitle

$\vec{n}.\vec{n}=1$\\

$nx^{2}+ny^{2}+n_{3}^{2}=1$\\

$N(\hat{e_{1}}, \hat{e_{2}}, \cdots \hat{e_{N}})$\\

m 3D \\

$\hat{e_{1}} \hat{e_{2}}, \hspace{2mm} \hat{e_{1}}\hat{e_{3}}, \hspace{2mm} \hat{e_{2}}\hat{e_{3}} \rightarrow \mbox{number of planes}$\\

number of planes $= \frac{N(N-1)}{2}=NC_{2}$\\

$s=\begin{pmatrix}
\cos \phi & \sin \phi\\
- \sin \phi & \cos \phi
\end{pmatrix}$\\

As $\phi
 \rightarrow \delta \phi$\\
 
 so, $s=\begin{pmatrix}
 1 & \delta \phi\\
 -\delta \phi & 1
 \end{pmatrix}$\\
 
 $s=\prod + T$ \hspace{5mm} $\mbox{So, } T=\begin{pmatrix}
 0 & \delta \phi\\
 -\delta \phi & 0
 \end{pmatrix}$\\
 
 $T^{\sim} =-T$\\
 
 $S=\prod + T$ \hspace{5mm} $S^{T}=\prod + T^{T}$\\
 
 $S^{T}S=\prod$ \hspace{5mm} to be orthogonal\\
 
 So, $S^{T}S=(\prod +T)(\prod + T^{T})=\prod$\\
 
 So, $\prod + T^{T} + T=\prod$ \hspace{5mm} "Working only on first order (m) this infinitesimal rotation"\\
 
 How many independent parameters in $N \times N$ show symmetries $\frac{N(N-1)}{2}$\\
 
 "Applying rotation n times"\\
 $S^{n}=(\prod + T)^{n}$ \hspace{5mm} , $T=\begin{pmatrix}
 0 & \delta \phi\\
 -\delta \phi & 0
 \end{pmatrix}$\\
 
 $\phi = n \delta \phi$, \hspace{5mm}, $\delta \phi=\frac{\phi}{n}$\\
 
$$\lim_{n \to \infty} \left(\prod +\frac{\phi}{n}R \right)^{n} \hspace{1cm} ,R=\begin{pmatrix}
0 & 1\\
-1 & 0
\end{pmatrix}$$\\

$$S=\lim_{n\to\infty} \left(\prod +T \right)^{n}\hspace{1cm} , T=\delta \phi R =\frac{\phi}{n}R$$\\

\begin{center}
\includegraphics[scale=.5]{3.png}
\end{center}
 
\begin{align*}
\vec{r} & =x\hat{e_{x}}+y\hat{e_{y}}\\
x' & =\cos \phi x+ \sin \phi y\\
y' & = \cos \phi y - \sin \phi x\\
\vec{r} & =e^{i \theta}\\
\end{align*} 

$\begin{pmatrix}
\sin \phi & \cos \phi & 0\\
0 & 0 & 1\\
\cos \phi & \sin \phi & 0
\end{pmatrix} \rightarrow \begin{pmatrix}
\cos \phi & 0 & \sin \phi\\
0 & 1 & 0\\
\sin \phi & 0 & \cos \phi 
\end{pmatrix}$ \hspace{5mm} Euler Rotations\\
\vspace{5mm}

\begin{center}
\includegraphics[scale=0.3]{4.png}
\includegraphics[scale=0.3]{5.png}
\end{center}

\newpage


\begin{enumerate}

\item $S_{1}(\alpha)= \mbox{rotation around  $\hat{e_{3}}$ by $\alpha$}$\\

$0 < \alpha \le 2\pi$

$\begin{pmatrix}
\hat{e'_{1}}\\
\hat{e'_{2}}\\
\hat{e'_{3}}
\end{pmatrix}=\begin{pmatrix}
\cos \alpha & \sin \alpha & 0\\
-\sin \alpha & \cos \alpha & 0\\
0 & 0 & 1
\end{pmatrix} \begin{pmatrix}
\hat{e_{1}}\\
\hat{e_{2}}\\
\hat{e_{3}}
\end{pmatrix}
$

\item $S_{2}(\beta)= \mbox{rotation around  $\hat{e_{2}}$ counter clockwise by $\beta$}$\\

$0 \le \beta <\pi$

$\begin{pmatrix}
\hat{e''_{1}}\\
\hat{e''_{2}}\\
\hat{e''_{3}}
\end{pmatrix}
=\begin{pmatrix}
\cos \beta&0&-\sin \beta\\
0&1&0\\
\sin \beta&0&\cos \beta
\end{pmatrix}
\begin{pmatrix}
\hat{e_{1}}\\
\hat{e_{2}}\\
\hat{e_{3}}
\end{pmatrix}$\\

\item $S_{3}(\gamma)= \mbox{rotation around $\hat{e''_{3}}$ by $\gamma$ counter clockwise}$\\

$0 \le \gamma <2\pi$\\

$\begin{pmatrix}
\hat{e''_{1}}\\
\hat{e''_{2}}\\
\hat{e''_{3}}
\end{pmatrix}=\begin{pmatrix}
\cos \gamma&\sin \gamma&0\\
-\sin \gamma& \cos \gamma&0\\
0&0&1
\end{pmatrix}\begin{pmatrix}
\hat{e_{1}}\\
\hat{e_{2}}\\
\hat{e_{3}}
\end{pmatrix}$\\

\begin{center}
\includegraphics[scale=0.5]{6.png}
\end{center}

\end{enumerate}

 \paragraph*{Differential Vector Operator:}
 
$\vec{r}(t)=x(t)\hat{i}+y(t)\hat{j}+z(t)\hat{k}$\\

$\frac{\vec{dr}(t)}{dt}=\frac{dx(t)}{dt}\hat{i}+\frac{dy(t)}{dt}\hat{j}+\frac{dz(t)}{dt}\hat{k}   $ \\

\begin{center}
\includegraphics[scale=0.5]{7.png}
\end{center}
 
 $\frac{\vec{dr}}{dt}=\lim_{\Delta t \to 0}\frac{\vec{\Delta r}}{\Delta t}$
 
 \newpage
 
 $\phi (x,y,z)=$ Scalar Field\\
 
 $\vec{\epsilon}=\epsilon_{x}(x,y,z)\hat{i}+\epsilon_{y}(x,y,z)\hat{j}+\epsilon_{z}(x,y,z)\hat{k}$\\
 
 $d \phi= \frac{\partial \phi}{\partial x_{1}}dx_{1}+\frac{\partial \phi}{\partial x_{2}}dx_{2}+\frac{\partial \phi}{\partial x_{3}}dx_{3}$\\
 
 $\vec{\nabla} \phi=\begin{pmatrix}
 \frac{\partial \phi}{\partial x_{1}}\\
 \frac{\partial \phi}{\partial x_{2}}\\
 \frac{\partial \phi}{\partial x_{3}}
 \end{pmatrix}$ \hspace{5mm} ,$\vec{dr}=\begin{pmatrix}
 dx_{1}\\
 dx_{2}\\
 dx_{3}
 \end{pmatrix}$\\
 
 \vspace{1cm}
 
 \begin{align*}
 \Delta \nabla \phi & =\begin{pmatrix}
\frac{\partial x_{1}}{\partial x'_{1}}&
\frac{\partial x_{2}}{\partial x'_{1}}&
\frac{\partial x_{3}}{\partial x'_{1}}\\
\frac{\partial x_{1}}{\partial x'_{2}}&\frac{\partial x_{2}}{\partial x'_{2}}&\frac{\partial x_{3}}{\partial x'_{3}}\\
\frac{\partial x_{1}}{\partial x'_{3}}&\frac{\partial x_{2}}{\partial x'_{3}}&\frac{\partial x_{3}}{\partial x'_{3}}
 \end{pmatrix}
 \begin{pmatrix}
 \frac{\partial \phi}{\partial x_{1}}\\
 \frac{\partial \phi}{\partial x_{2}}\\
 \frac{\partial \phi}{\partial x_{3}}
 \end{pmatrix}\\
 & =\begin{pmatrix}
 \sum_{\nu =1}^{3}&\frac{\partial \lambda_{\nu}}{\partial x'_{1}}&\frac{\partial \phi}{\partial x_{\nu}}\\
\sum_{\mu =1}^{3}&\frac{\partial x_{\nu}}{\partial \lambda'_{2}}&\frac{\partial \phi}{\partial x_{\mu}}\\ 
\sum_{\mu =1}^{3}&\frac{\partial x_{\nu}}{\partial x'_{3}}&\frac{\partial \phi}{\partial x_{\nu}}
\end{pmatrix}\\
& =\begin{pmatrix}
\frac{\partial \phi}{\partial x'_{1}}\\
\frac{\partial \phi}{\partial x'_{2}}\\
\frac{\partial \phi}{\partial x'_{3}}
\end{pmatrix}
\end{align*}

\begin{center}
\includegraphics[scale=0.5]{8.png}
\end{center}
\newpage

$\left(\nabla \phi\right)_{i}=\sum_{j=1}^{3} \delta_{ij}(\nabla \phi)_{j}$\\

$\nabla' \phi=\frac{\partial \phi}{\partial x_{1}}\hat{_{1}}+\frac{\partial \phi}{\partial x_{2}}\hat{_{2}}+\frac{\partial \phi}{\partial x_{3}}\hat{_{3}}$\\

A gradient of a pseudo scalar is a pseudo vector.\\

A gradient of a real scalar is a polar vector.\\

$|\vec{\nabla} \phi|=\sqrt{\left(\frac{\partial \phi}{\partial x_{1}}\right)^{2}+\left(\frac{\partial \phi}{\partial x_{2}}\right)^{2}+\left(\frac{\partial \phi}{\partial x_{3}}\right)^{2}}$\\

$d \phi = \vec{\nabla} \phi
. \vec{dr}=|\vec{\nabla} \phi||\vec{dr}|\cos \theta$ \hspace{1cm} Max. change of $\phi \rightarrow \theta =0$\\

$d \phi_{max}=|\vec{\nabla \phi}||\vec{dr}|=|\vec{\nabla} \phi||\vec{dr}|\cos \theta=0$\\

As $\theta =\frac{\pi}{2}$\\

so, $\phi= \mbox{constant}
\rightarrow d \phi=0$\\
 

\end{document}